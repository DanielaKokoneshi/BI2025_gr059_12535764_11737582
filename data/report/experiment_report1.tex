\documentclass[sigconf]{acmart}

\AtBeginDocument{ \providecommand\BibTeX{ Bib\TeX } }
\setcopyright{acmlicensed}
\copyrightyear{2025}
\acmYear{2025}
\acmDOI{XXXXXXX.XXXXXXX}

\acmConference[BI 2025]{Business Intelligence}{-}{-}

\begin{document}

\title{BI2025 Experiment Report - Group 059}
%% ---Authors: Dynamically added ---

          \author{Daniela Kokoneshi}
          \authornote{Student A, Matr.Nr.: 12535764}
          \affiliation{
            \institution{TU Wien}
            \country{Austria}
          }
          
          \author{Keisi Cela}
          \authornote{Student B, Matr.Nr.: 11737582}
          \affiliation{
            \institution{TU Wien}
            \country{Austria}
          }
          

\begin{abstract}
  This report documents the machine learning experiment for Group 059, following the CRISP-DM process model.
\end{abstract}

\ccsdesc[500]{Computing methodologies~Machine learning}
\keywords{CRISP-DM, Provenance, Knowledge Graph, Machine Learning}

\maketitle

%% --- 1. Business Understanding ---
\section{Business Understanding}

\subsection{Data Source and Scenario}
The colloges\_usnews dataset used for this poject is gotten from OpenML.
It is a real-world dataset, from the American Statistical Association's
1995 Data Analysis Exposition,based on insitutional data published in U.S
News and World Report`s Guide to America`s Best Colleges (1993-1994).
It contains information for 1,302 U.S colleges and universities,
including variables related to admission selectivity (SAT/ACT scores,
application acceptance), institutional characterisitcs (public/private,
tuition, spending), student composition,faculty qualification and outcomes
such as graduation rate. In total the dataset has a total of 35 attributes
with most of them being numeric, and some being categorical such as state,
public/private. The dataset contains missing values, particularly in the
test score variables making it a realistic analysis.

A suitable business scenraio for this dataset would be a university
administration seeking to understand what factors influence graduation rates.
By applying data analytics processess to the dataset, the institution can
identify patterns and understand what factors influence a higher graduation rate.
This analysis can support strategic planning, rescource allocation and performance
benchmarking.

\subsection{Business Objectives}
The main objective of this analysis is to define which of the variables
have a higher impact on the graduation rate of students. By identifying
these factors the institutions have the necessary information to improve
students sucess. Other objectives include providing universities with
benchmarks against similar institution, identifying what aspects would
have the biggest turn on investments and understaning how admission selectivity
or student preparedness relate to graduation outcomes.

All these objectives support organizations in making informed strategic decisions.
\subsection{Business Success Criteria}
For this project to be called successful, it should produce  precise
 and easily interpretable insights regarding the key factors that influence
 the graduation rate as well as a model that accuratly predicts
the graduation rate of an institution based on the provided features. Another
measure of sucess for this analysis would be that the derived recommendations
are actionable, such as identifying which variables universities should
influence to improve outcomes.

Additionally, the results should be presented in a way  that universities
can easily use the results for benchmarking, this way they can easily
compare their performance against other educational institutions.
\subsection{Data Mining Goals}
The main data mining goal is to build a predictive model that estimates
an institution`s graduation rate based on the available institutional,
financial and student-related variables in the dataset. The model should
provide accurate predictions, as well as highlight which features
contribute most strongly to higher or lower graduation rates. Another
goal is to perform exploratory analysis to better understand the relationshipps
between variables such as tuition, selectivity, spending per student,
or faculty resources and the graduation rate.This supports the business
objective of identifying concrete levers that universities can influence
 to improve student successs.
\subsection{Data Mining Success Criteria}
From a data mining perspective, the project is considered successful if
we can train a regression model that generalizes well and explains an important
 part of the variance in graduation rates. We are trying to have a model that
 achieves a high coefficient of determination and a low prediction error
 when testing via cross-validation.In addition, the learned model and derived
 feature importance measures should be interpretable for non-technical stakeholders.
 Key factors influencing graduation rate should be distinctly identifiable
 (similar across different validation splits), so that they can be translated
 into recommendations and benchmarking indicators for universities.
\subsection{AI Risk Aspects}
This project uses institutional-level, historical data about colleges and
universities. While the dataset does not contain personal student identifiers,
there are still some AI-related risks to consider. First, the data may reflect
existing structural inequalities (between public and private institutions or
between different regions). A model trained on such data could unintentionally
reinforce these patterns if its predictions are used for funding decisions,
 rankings or policy making without careful interpretation. Second, there is
 a risk of over-interpreting correlations as causal effects and using the
 model as a black-box decision tool rather than as analytic support. To handle
  these risks, the focus of the project is on transparency and interpretability
  (using explainable models and feature importance analysis), precisely
  communicating the limitations of the data and the model, and treating the
  results as decision-support for experts rather than automatic decision-making.

%% --- 2. Data Understanding ---
\section{Data Understanding}

\subsection{Dataset Description}
\textbf{Dataset Description:} {du_description}

This dataset includes a variety of institutional, financial, and student outcome variables. 
It was obtained from OpenML and contains all attributes present in the original ARFF file.

\subsection{Raw Data Features}

The following features were identified and documented in the knowledge graph:

\begin{table}[h]
  \caption{Raw Data Features}
  \label{tab:features}
  \begin{tabular}{lp{0.2\linewidth}p{0.4\linewidth}}
    \toprule
    \textbf{Feature Name} & \textbf{Data Type} & \textbf{Description} \\
    \midrule
    {du_table_rows}
    \bottomrule
  \end{tabular}
\end{table}

\subsection{Descriptive Statistics}
The descriptive statistics for all attributes were computed during the Data Understanding phase.  
Summary from the knowledge graph:

\begin{quote}
{du_stats_comment}
\end{quote}

\subsection{Data Quality Assessment}
A systematic evaluation of data quality was conducted by checking for missing values and investigating outlier distributions.

\subsubsection*{Missing Values}
\begin{quote}
{du_missing_comment}
\end{quote}

\subsubsection*{Outliers (IQR Method)}
\begin{quote}
{du_outliers_comment}
\end{quote}

\subsection{Visual Exploration}
Simple visual analytics were generated to understand the distribution of the target variable and its relationships with selected numeric features.  
Summary from the visual exploration activity:

\begin{quote}
{du_visual_comment}
\end{quote}

\subsection{Ethical and Sensitive Aspects (2e)}
\begin{quote}
{du_ethics_comment}
\end{quote}

\subsection{Risks and Open Questions (2f)}
\begin{quote}
{du_risks_comment}
\end{quote}

\subsection{Planned Data Preparation Actions (2g)}
\begin{quote}
{du_prep_comment}
\end{quote}

%% --- 3. Data Preparation ---
\section{Data Preparation}
\subsection{Data Cleaning}
Describe your Data preparation steps here and include respective graph data.


%% --- 4. Modeling ---
\section{Modeling}

\subsection{Hyperparameter Configuration}
The model was trained using the following hyperparameter settings:

\begin{table}[h]
  \caption{Hyperparameter Settings}
  \label{tab:hyperparams}
  \begin{tabular}{lp{0.4\linewidth}l}
    \toprule
    \textbf{Parameter} & \textbf{Description} & \textbf{Value} \\
    \midrule
    
    \bottomrule
  \end{tabular}
\end{table}

\subsection{Training Run}
A training run was executed with the following characteristics:
\begin{itemize}
    \item \textbf{Algorithm:} 
    \item \textbf{Start Time:} 
    \item \textbf{End Time:} 
    \item \textbf{Result:}  = 
\end{itemize}

%% --- 5. Evaluation ---
\section{Evaluation}

%% --- 6. Deployment ---
\section{Deployment}

\section{Conclusion}

\end{document}
